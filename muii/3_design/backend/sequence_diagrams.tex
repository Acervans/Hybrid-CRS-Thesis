To illustrate the dynamic behavior of the system, sequence diagrams are used to model the flow of interactions between different components for key use cases. The following diagrams detail three of the most critical processes in the platform.

\subsubsection{Agent Creation Sequence}
The creation of a new recommender agent is a complex, asynchronous process that involves nearly every component of the system. The sequence, shown in Figure~\ref{FIG:SEQ_CREATE_AGENT}, begins when a user submits the dataset files and configuration through the frontend. The backend \acs{api} receives this request and initiates the data processing pipeline, which in turn interacts with the data stores and recommendation module to build and train the agent.

\begin{figure}[Sequence Diagram for Agent Creation]{FIG:SEQ_CREATE_AGENT}{Sequence Diagram for Agent Creation.}
    \centering
    % \includegraphics[width=\textwidth]{sequence_create.png}
    \includesvg[inkscapelatex=false,width=\textwidth]{placeholder}
\end{figure}

\subsubsection{Agent Chat Interaction Sequence}
A typical agent chat interaction involves a conversation between the user, the \ac{llm}, and the underlying recommendation engines. As shown in Figure~\ref{FIG:SEQ_AGENT_CHAT}, a user's message is sent to the backend, which forwards it to the LlamaIndex workflow. The workflow may use Function Calling to invoke the recommendation module (either the graph-based or expert model) to retrieve a list of recommendations with accompanying explanations, which are then formatted into a natural language response by the \ac{llm} and streamed back to the user.

\begin{figure}[Sequence Diagram for an Agent Chat Interaction]{FIG:SEQ_AGENT_CHAT}{Sequence Diagram for an Agent Chat Interaction.}
    \centering
    % \includegraphics[width=0.8\textwidth]{sequence_chat.png}
    \includesvg[inkscapelatex=false,width=0.8\textwidth]{placeholder}
\end{figure}

\subsubsection{Open Chat Interaction Sequence}
This interaction describes the open chat functionality, where the user interacts freely with any \ac{llm} available in the platform. The sequence diagram in Figure~\ref{FIG:SEQ_OPEN_CHAT} illustrates how the user sends a chat request to the backend, which then proxies it to the Ollama service. The model generates a response and streams the reply back to the user through the backend server.

\begin{figure}[Sequence Diagram for an Open Chat Interaction]{FIG:SEQ_OPEN_CHAT}{Sequence Diagram for an Open Chat Interaction.}
    \centering
    % \includegraphics[width=0.8\textwidth]{sequence_chat.png}
    \includesvg[inkscapelatex=false,width=0.8\textwidth]{placeholder}
\end{figure}