The primary objective of this Master's Thesis is the design and implementation of a scalable, modular, and reproducible platform (\acl{crs}) for building and deploying hybrid conversational recommender agents. The focus is centered on the engineering and architectural challenges inherent in creating a robust, full-stack application. The specific objectives are detailed as follows.

\begin{objetive}
    \item \textbf{Detailed System Design:} To define and document a resilient and scalable architecture that integrates multiple components, including a frontend user interface, a backend \acs{api} with all the application logic, a \ac{llm} service, and data storage solutions. This architecture will be designed for modularity to facilitate independent development, testing, and deployment of each component.

    \item \textbf{High-Performing Backend Service:} To develop a backend using the FastAPI framework that serves as the backbone of the platform. This includes creating \acs{api} endpoints for managing the lifecycle of recommender agents, handling user interactions, processing data, starting and interacting with conversational workflows and proxying requests to the \ac{llm}.

    \item \textbf{Automated Data Processing Pipeline:} To build a system capable of automatically ingesting, cleaning, and structuring user-provided datasets. This pipeline will standardize heterogeneous data files into a format suitable for both the graph-based recommender and the expert \acs{rs} model, ensuring data integrity and consistency.

    \item \textbf{Responsive Frontend Application:} To build a modern \ac{pwa} using the Next.js framework. This interface will provide users with a comprehensive dashboard to create and manage their recommender agents, as well as an intuitive chat interface to interact with them. Emphasis will be placed on usability, accessibility, and providing a seamless user experience.

    \item \textbf{Scalability and Reproducibility with Containerization:} To utilize Docker to containerize each component of the system (backend, database, \ac{llm} service). This approach will ensure that the entire platform is portable, easy to deploy, and horizontally scalable, adhering to modern DevOps practices.

    \item \textbf{System Testing and Evaluation:} To perform comprehensive testing, including unit and integration tests for backend logic, \acs{api} load testing to measure performance under stress, usability testing with real users, and a comparative analysis of the platform's contributions against established benchmarks or similar systems where applicable.
\end{objetive}

\textcolor{red}{I don't know if in this section or at the beginning of this chapter, but you should mention the other thesis. Perhaps here it is a good moment. You say 'the focus is centered on the engineering...', and then you can mention that another Master thesis is devoted to the research component and evaluation of the conversational and recommendation parts.}
