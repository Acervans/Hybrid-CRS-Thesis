The primary objective of this project is to explore and understand the effectiveness of applying sentiment attributes in the domain of music recommendation, by investigating the role of mood in music choice and its representation of user preferences. This will be accomplished through the collection of a new dataset using Last.fm's \ac{api}, followed by the generation of sentiment attributes using a text sentiment analyzer. This dataset will be used to train different models and to test their integration of sentiment by evaluating recommendation accuracy. For this goal, context-aware models are of interest, as these algorithms are able to seamlessly embed additional attributes into the recommendation process. Finally, a web application will be developed as a platform to showcase the tools used throughout the project, along with the final recommendation models.

However, it is worth mentioning that the real value of the thesis lies in the process rather than the outcome since, throughout its making, it entailed the deepening of topics such as natural language processing, music recommendation and web scraping, which were not touched upon thus far; together with the opportunity to put into practice all the knowledge acquired during these years, namely software engineering, web development, database management and machine learning.