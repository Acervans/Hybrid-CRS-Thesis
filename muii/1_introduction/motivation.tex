The recent and rapid proliferation of \acp{llm} has sparked a transformative wave across numerous technological domains. These models, exemplified by systems like OpenAI's \ac{gpt} \cite{CHATGPT} and Google's Gemini \cite{GEMINI}, have demonstrated remarkable capabilities in understanding and generating human-like text, leading to a surge in the research and development of conversational systems for a wide array of applications. While rule-based and pattern-matching conversational recommenders have existed for some time \cite{SOTA-CRS}, the integration of the contextual understanding of \acp{llm} into the specialized field of \acl{rs}s is still a relatively unexplored topic \cite{SOTA-CRS-LLM}.

This integration, however, is not a trivial task. It presents significant engineering challenges related to system architecture, data management, scalability, and the seamless fusion of conversational interfaces with complex recommendation algorithms. The challenge lies not only in leveraging the conversational power of \acp{llm} but in building a robust, end-to-end platform that can be easily configured, extended, and deployed.

This project is motivated by the opportunity to address these challenges with a practical approach. The main goal is to harness the potential of \acp{llm} to design and implement a versatile platform for creating \acp{crs}. This platform will be engineered to be domain-agnostic and automatically extensible with external datasets. By employing modern software engineering practices and technologies, this work aims to create a system that is not only powerful in its functionality---offering natural conversations to elicit user preferences \cite[Conversational Preference Elicitation]{CHAPTER:RS-HANDBOOK-NLP} and generating explained recommendations---but also scalable, reproducible, and maintainable. The emphasis is on the practical engineering aspects required to build a sophisticated, full-stack application that translates the theoretical potential of \acp{llm} into a practical solution for \acl{rs}s.

Despite the existence of conversational libraries that have become quite common nowadays, this project was built as a standalone system to avoid relying on tools that evolve quickly and may become obsolete. It also provided an opportunity to understand how such conversational agents work by implementing the entire pipeline independently, allowing full control over its architecture and behavior.

\textcolor{red}{Somewhere in this section, you should motivate or mention that this project is defined as a standalone project, and it does not use any conversational library that are so common nowadays. One possible reason would be to not depend on libraries that advance too quickly and could be easily outdated, another reason could be simply that you wanted to learn to create one of these systems from scratch}
