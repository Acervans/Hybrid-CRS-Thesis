This document is structured into six chapters, designed to provide a comprehensive overview of the project, from its design to its implementation and evaluation.

\begin{description}
    \item[Chapter 1 --- Introduction] This initial chapter provides the motivation for the project, outlining the current landscape of \acp{llm} and \acp{crs} and identifying the engineering challenges this thesis aims to address. It also defines the main objectives of the work and presents this overview of the document's structure.

    \item[Chapter 2 --- State of the Art] This chapter reviews the key technologies and existing research that form the foundation of this project. It provides an overview of \acp{llm} and their associated frameworks, discusses their application in \acl{rs}s, and examines technologies required for data management and scalability, such as vector and graph databases. Finally, it reviews the web application frameworks used in the implementation.

    \item[Chapter 3 --- System Design] This chapter presents a detailed blueprint of the platform's architecture. It begins with an analysis of the system's functional and non-functional requirements. It then describes the high-level architecture, including use case diagrams, and delves into the specific design of the backend and frontend components, detailing module definitions, sequence diagrams, database schemas, and data flowcharts.

    \item[Chapter 4 --- Implementation] This chapter details the technical execution of the design presented in the previous chapter. It covers the development of the backend services, including the integration of the \ac{llm} and the \acs{rs} models, and the creation of the frontend \ac{pwa}. It also describes the deployment strategy, encompassing the frontend deployment, the use of Docker for containerization, the \ac{cicd} pipeline, and the tunneling solution for exposing the backend service.

    \item[Chapter 5 --- Testing and Evaluation] This chapter focuses on the validation and assessment of the implemented platform. It describes the testing environment and methodologies, covering unit and integration testing for all major components, performance and load testing of the \acs{api}, and a high-level summary of the usability testing results. A comparative analysis of the platform's contributions is also discussed.

    \item[Chapter 6 --- Conclusions and Future Work] The final chapter summarizes the key achievements and contributions of this thesis, reflecting on how the initial objectives were met. It also discusses the limitations of the current work and proposes potential avenues for future research and development, suggesting ways to extend and improve the platform.
\end{description}