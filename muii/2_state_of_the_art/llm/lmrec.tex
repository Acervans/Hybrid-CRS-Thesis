The application of \acp{llm} to the domain of \acl{rs}s is a growing field of research that promises to redefine how users interact with recommendation services \cite{SOTA-CRS-LLM}. Traditionally, users interact with \acl{rs}s through rigid interfaces. By integrating \acp{llm}, it is possible to create dynamic, conversational experiences where users can express their preferences in natural language, ask for clarifications, and receive recommendations that are not only relevant but also contextually explainable by the language model.

The primary role of the \ac{llm} in this context is to act as a natural language interface between the user and the underlying recommendation engine. It can parse user queries, maintain conversational context, and format the output from the recommender into a coherent and helpful response. This thesis focuses on the engineering aspects of building a platform that facilitates this integration, creating a scalable system where conversational agents can be powered by expert recommender models. The specific research questions regarding the effectiveness of different conversational strategies and their impact on user satisfaction, recommendation quality and transparency are explored in greater detail in the complementary research-focused thesis \cite{MUI2ICSI:SOTA}.