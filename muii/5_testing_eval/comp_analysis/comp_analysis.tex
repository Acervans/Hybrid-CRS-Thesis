To contextualize the contributions of this thesis, this section provides a comparative analysis, positioning the developed platform against the state-of-the-art commercial and open-source systems. The comparison is structured around two main areas: the user-facing interface and the underlying backend architecture.

\subsubsection{User Interface and Interaction Paradigm}

The platform's frontend was designed to incorporate and synthesize several advanced paradigms observed in leading conversational platforms.

\begin{compactitem}[\textbullet]
    \item \textbf{Core Chat Interface:} The fundamental \acs{ui}, with its sidebar for chat history and a central message viewport, adopts the foundational design paradigm popularized by \textbf{ChatGPT} \cite{CHATGPT}. Like ChatGPT, it supports file uploads and advanced interaction beyond simple text.
    
    \item \textbf{Self-Hosted Philosophy:} Architecturally, the platform aligns closely with the ethos of \textbf{Open WebUI} \cite{OPENWEBUI}. By using Docker and supporting locally-hosted models via Ollama, it provides a self-hostable, privacy-preserving alternative to closed, proprietary systems.

    \item \textbf{User Control and Context Scoping:} The ability for a user to create and chat with distinct agents, each trained on a specific dataset, implements a core principle of user control. This is conceptually similar to \textbf{Perplexity AI}'s ``Spaces'' \cite{PERPLEXITY} or \textbf{Google Gemini}'s ``Gems'' feature, which allows users to create personalized agents with specific contexts, leading to more tailored results.

    \item \textbf{Collaborative Workspace:} Critically, the platform moves beyond a simple chat interface and successfully implements a Collaborative Workspace. It features a dedicated agent creation page that serves as a dataset management workspace, allowing users to upload and configure the data that powers their agents, before sharing them with others. This directly aligns with \textbf{Claude Artifacts}, which allows users to share ideas created with Claude.
\end{compactitem}

\subsubsection{Backend Architecture and Development Framework}

The backend architecture and choice of development framework align with modern best practices for building \ac{llm}-powered applications.

\begin{compactitem}[\textbullet]
    \item \textbf{Agent Framework:} The implementation of the conversational agent uses \textbf{LlamaIndex}, a state-of-the-art, data-centric framework that excels at creating knowledge-intensive applications. This is a more flexible and powerful approach for a platform-oriented project compared to using more rigid, specialized academic toolkits like CRSLab \cite{CRSLAB}.

    \item \textbf{Modular Design:} The system's architecture is represented by the modular ``Text-in, Text-out'' philosophy seen in toolkits like \textbf{RecWizard} \cite{RECWIZARD}. Instead of using a simple \ac{rag} pipeline, the project uses a LlamaIndex workflow to orchestrate a hybrid recommendation strategy. The conversational agent does not retrieve recommendations directly; rather, it uses function calling to intelligently invoke external, specialized tools: a pre-trained \texttt{EASER} expert model managed by RecBole, and a real-time, explainable recommender powered by the \textbf{FalkorDB} graph database. This multi-component approach demonstrates an advanced and robust implementation of modern conversational \acs{ai} concepts.
\end{compactitem}