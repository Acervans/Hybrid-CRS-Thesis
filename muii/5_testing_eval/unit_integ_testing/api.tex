To validate the \acs{api} layer, integration tests were written using FastAPI's \texttt{TestClient}, which simulates HTTP requests to the application without needing to run a live server. The tests, located in \texttt{test\_api.py}, make use of the \texttt{pytest-mock} library to isolate the \acs{api} from its downstream dependencies.

This mocking strategy is paramount for focused testing. For example, in \texttt{test\_create\_agent}, all backend logic--including calls to Supabase, FalkorDB, and the RecBole training functions--is mocked. This allows the test to verify that the \texttt{/create-agent} endpoint correctly handles multipart form data, parses the request body, and calls the appropriate backend functions, without executing the time-consuming processes of data ingestion and model training. Similarly, the JWT authentication middleware is mocked in all tests to bypass the need for a valid token, allowing the focus to remain on the endpoint's logic.