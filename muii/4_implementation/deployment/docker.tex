Containerization with Docker \cite{DOCKER} is the cornerstone of the platform's deployment strategy, ensuring consistency, portability, and reproducibility across development and production environments. Docker also improves security and scalability by isolating each service in its own container, reducing dependency conflicts and limiting the spread of vulnerabilities. The entire backend stack is defined as a set of services in a \texttt{docker-compose.yaml} file, which orchestrates the FastAPI application, the Ollama \ac{llm} server, and the FalkorDB database instance (as well as the Next.js frontend application during development).

The backend's \texttt{Dockerfile} utilizes multi-stage builds to create an optimized and lightweight final image. A \textbf{deps} stage is used to install Python dependencies, and the final \textbf{runtime} stage copies only the necessary application code and installed packages. This practice leverages Docker's layer caching, which automatically triggers a rebuild of the image only when source code or dependencies are modified. Furthermore, remote service images like Ollama can be easily kept up-to-date by pulling the newest versions from image repositories (e.g., Docker Hub), ensuring the system runs on the latest stable versions.
