Due to the significant computational and memory requirements of the backend services (particularly the \ac{llm}, graph database and recommendation models) and the cost limitations of free-tier cloud hosting, the backend stack is run on local hardware. To make these local services accessible to the publicly deployed frontend on Vercel, a secure tunneling solution was implemented.

The open-source tool Zrok \cite{ZROK} was chosen for this purpose. Zrok provides a secure way to create a public URL for proxying requests to a local web service. It was preferred over alternatives like Ngrok due to its more generous free-tier resources. The process involves a few simple steps: first, enabling the Zrok environment with an account token; second, reserving a unique and persistent public name (and domain) for the backend service; and finally, initiating the share, which creates a secure tunnel from the public Zrok URL to the locally running FastAPI server at a specific port.