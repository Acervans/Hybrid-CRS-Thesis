The project adopts a pragmatic and differentiated approach to \acs{cicd} for the frontend and backend.

For the \textbf{frontend} a full \acs{cicd} pipeline is implemented through its integration with Vercel. On every push to the main branch, Vercel's pipeline automatically builds and deploys the application. Code quality is enforced locally before commits via a Husky pre-commit hook that runs ESLint and Prettier to ensure consistent code style and linting.

For the \textbf{backend}, a decision was made not to implement a fully automated \acs{cicd} pipeline. As a solo project, the overhead of configuring and maintaining such a pipeline outweighed its benefits. Instead, code quality and integration are ensured through robust pre-commit hooks. These hooks automatically format Python code with \texttt{black}, and run the entire test suite using \texttt{pytest} from the \texttt{tests/} folder. This local CI process is sufficient for maintaining code quality. Furthermore, the backend's dependency on stateful services like FalkorDB and a running Ollama instance with a specific model loaded makes automated testing in a standard CI runner environment prohibitively complex and costly.

\textcolor{red}{a qué te refieres con 'solo project'? me lo puedo imaginar, pero por lo que he buscado, no es una expresión que se use}