All communication between the frontend and the backend is handled through the RESTful \acs{api} exposed by the FastAPI server. The frontend implements a set of client-side functions, primarily located in \texttt{src/lib/api.ts}, which encapsulate the logic for making \textit{fetch} requests to the various backend endpoints. Also, all the interactions with the Supabase database tables, whether to add, edit or delete records, are implemented in \texttt{src/lib/supabase/client.ts}, clearly separating backend from Supabase functionalities (mostly used for metadata).

Notably, the integration is found in the \texttt{AssistantProvider}. For the Open Chat view, it uses the \texttt{streamChat} function to send user messages to the backend's Ollama proxy endpoint, and receive a streamed response. For agent-specific conversations, the \texttt{startWorkflowApi} function is used to initiate a new session. The \texttt{WorkflowContext} then manages the real-time event stream from the server, parsing events to display streamed text or new recommendations.

The Agent Hub view also demonstrates this integration. When a user performs an action like deleting or retraining an agent, it calls the corresponding functions (\texttt{deleteAgent}, \texttt{retrainAgent}), which handle the authenticated requests to the backend asynchronously, providing a seamless user experience while the backend performs the necessary tasks.