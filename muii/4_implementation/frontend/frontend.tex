The frontend of the platform was implemented as a modern, responsive Progressive Web Application using the Next.js 15 framework and TypeScript. The primary goal was to create a clean, intuitive, and highly interactive user interface that effectively abstracts the complexity of the underlying backend services. In the main repository, the frontend code is located in the \texttt{frontend} directory as a submodule, which allows for independent development and deployment processes.

Development followed best practices for code quality and consistency. The project was configured to use Husky, a tool that manages Git hooks, to run automated scripts before each commit. Specifically, a pre-commit hook was set up to trigger Prettier for code formatting and ESLint for code linting. This automated process ensures that all contributions to the codebase adhere to a consistent style guide and are free from common syntactical errors, significantly improving code readability and maintainability.