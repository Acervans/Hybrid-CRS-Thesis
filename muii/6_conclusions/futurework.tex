\textcolor{red}{again: this is not from this project}

With reference to the topics discussed in this project, several aspects may be addressed hereafter. To begin with, all the tested recommendation models could be enhanced with exhaustive hyperparameter tuning, indispensable to achieve optimal results; and newer models or libraries could be explored to diversify the scope and types of recommendation algorithms.

To obtain real feedback, testing with external, real-world users could also be conducted, both on the performance and accuracy of the models, as well as the user experience of the developed web application.

Regarding the scraped dataset, a suggestion would be making it dynamic by allowing the addition of new users, which entails retraining the models. For this, the project might delve into the use of distributed or parallel computing with technologies like Spark, Hadoop, containerized environments or cloud services for efficient model deployment.

To evaluate the sentiment analyzer, formal accuracy testing would be an option to assess its real efficacy, and alternative sources for extracting sentiment attributes, such as lyrics from tracks, could be considered to achieve more representative and subjective text for sentiment analysis.

Last but not least, future research shall focus on broadening the understanding of \acs{vad} or different sentiment models, aiming to refine sentiment analysis in music recommendation or other domains, and potentially uncovering novel insights for the coming years.