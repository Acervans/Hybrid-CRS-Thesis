This Master's Thesis set out to address the engineering challenges of integrating modern \acp{llm} into a scalable and user-friendly platform or \acl{crs} for creating recommender agents. The primary contribution of this work is the successful design, implementation, and evaluation of a complete, end-to-end, full-stack platform that achieves this goal.

The project successfully met its core objectives. A modular, containerized architecture was designed and implemented, providing a clear separation between the Next.js frontend, the FastAPI backend, and the various \acs{ai} and data services. The backend provides a robust and high-performance \ac{api} that orchestrates the platform's complex domain logic, while the frontend delivers a modern, responsive user experience that incorporates state-of-the-art \acs{ui} paradigms like the ``Recommender Workspace''. An automated data processing pipeline was developed to handle the ingestion and preparation of user-provided datasets, and the entire system's portability and reproducibility are ensured through the use of Docker containers. Rigorous performance and unit testing have validated the platform's efficiency and reliability.

Ultimately, this project delivers a tangible and functional platform that serves as a strong foundation for both practical application and future research. It successfully bridges the gap between the theoretical potential of conversational \acs{ai} and the practical engineering required to build a deployable system, providing a valuable tool for anyone looking to create and interact with personalized, conversational recommender agents. 

For those interested in the more research-focused aspects of the project, the complementary thesis \cite{MUI2ICSI_THESIS} provides a detailed exploration of the conversational workflow, the underlying algorithms for recommendation and explanation, along with the usability evaluation's procedure and results.

The project repository can be found at \url{https://github.com/Acervans/Hybrid-CRS}. It contains all the code developed for the project, along with steps to set up the FastAPI backend, the Next.js frontend, and the necessary services.