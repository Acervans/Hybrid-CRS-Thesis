Los sistemas de recomendación musical desempeñan un papel fundamental al atender a diversas preferencias de usuarios y fomentar experiencias musicales personalizadas. De igual manera, el sentimiento puede influir profundamente en la música al dar forma a su expresión emocional y evocar estados de ánimo específicos en los oyentes. Dicho sentimiento se puede analizar mediante técnicas de procesamiento del lenguaje natural para medir emociones u opiniones expresadas en contenidos textuales, con la esperanza de aumentar la relevancia o significado al aplicarse en procesos de recomendación.

Este proyecto se adentra en el campo del análisis de sentimiento y su posible impacto en la recomendación musical, buscando mejorar los modelos de recomendación al incorporar atributos de sentimiento derivados de un conjunto de datos y un analizador de sentimientos obtenidos con este fin, con objeto de encontrar relaciones significativas entre las emociones y el gusto musical.

Como culmen de este trabajo, se desarrolla una aplicación web para mostrar las herramientas utilizadas y los modelos de mejor rendimiento. Esta plataforma ilustra el potencial de la recomendación consciente de emociones, ofreciendo una perspectiva sobre la dimensión de los sentimientos y su representación de preferencias personales.

\palabrasclave{Sistemas de recomendación, recomendación de música, análisis de sentimiento, scraping de datos, procesamiento del lenguaje natural}
