La reciente proliferación de los \textit{\acp{llm}} ha impulsado un interés significativo en el desarrollo de sistemas conversacionales avanzados. Sin embargo, la integración de estos modelos en plataformas de Sistemas de Recomendación que sean escalables, robustas y fáciles de usar sigue siendo un desafío complejo de ingeniería. Las soluciones actuales suelen carecer de un marco completo que simplifique el despliegue y asegure la escalabilidad en aplicaciones reales.

Este Trabajo de Fin de Máster trata de abordar este problema, detallando el diseño y la implementación de una plataforma modular y escalable para la creación de agentes conversacionales de recomendación híbrida. El objetivo principal es desarrollar una solución completa \textit{full-stack} que automatice todo el ciclo de vida de un agente conversacional, desde su creación y el procesamiento de datos hasta su despliegue e interacción con el usuario.

La plataforma está diseñada sobre una arquitectura contenerizada con Docker para garantizar modularidad, portabilidad y escalabilidad. El \textit{backend} se construye sobre FastAPI, un \textit{framework} de alto rendimiento que expone una \ac{api} RESTful para gestionar agentes, procesar conjuntos de datos y manejar la lógica conversacional. Para el \textit{frontend}, se desarrolló una \textit{\ac{pwa}} \textit{adaptativa} e instalable usando Next.js, que proporciona una interfaz intuitiva para que los usuarios administren y conversen con sus agentes de recomendación. La capa de datos del sistema utiliza Supabase, una base de datos relacional en la nube para almacenar metadatos de los agentes y las conversaciones; y FalkorDB como base de datos de grafos de alto rendimiento para modelar las relaciones entre usuarios e ítems, y para almacenar historiales de chat. Una característica destacable es el \textit{pipeline} de procesamiento de datos automatizado que se encarga de la ingesta, limpieza y estructuración de los conjuntos de datos subidos por el usuario.

El resultado final de este proyecto es una plataforma \textit{end-to-end} completamente funcional que permite a los usuarios crear, gestionar e interactuar de manera fluida con agentes de recomendación conversacionales personalizados. La robusta arquitectura prioriza la eficiencia, la escalabilidad y la reproducibilidad, proporcionando una base sólida para futuros desarrollos e investigaciones en la intersección de la IA conversacional y los sistemas de recomendación.

\palabrasclave{Sistema de Recomendación Conversacional, Modelos Extensos de Lenguaje, Desarrollo Full-stack, Arquitectura Escalable, Diseño de Sistemas, Bases de Datos de Grafos, FastAPI, Next.js, Docker}
