Music recommendation systems play a pivotal role in catering to diverse user preferences and fostering personalized listening experiences. Likewise, sentiment can profoundly influence music by shaping its emotional expression and evoking specific moods onto listeners. This sentiment may be analyzed through natural language processing techniques to gauge emotions or opinions expressed in textual content, hopefully increasing relevance or significance when applied to the recommendation process.

This project ventures into the field of sentiment analysis and its potential impact on music recommendation, attempting to enhance recommendation models by incorporating sentiment attributes derived from a manually retrieved dataset and a bespoke sentiment analyzer --- in pursuit of insightful correlations between emotion and musical taste.

As a culmination of this research endeavor, a web application is developed to showcase the applied tools and the finest-performing recommender models. This platform illustrates the potential of emotion-aware recommendation, offering a view into the dimension of sentiment and its representation of personal preferences.

\keywords{Recommender systems, music recommendation, sentiment analysis, data scraping, natural language processing}
