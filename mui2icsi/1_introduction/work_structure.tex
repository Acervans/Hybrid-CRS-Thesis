This thesis is organized into five chapters, structured to logically present the research from its conceptualization to its final conclusions.

\begin{compactitem}[\textbullet]
    \item \textbf{Chapter 2 - State of the Art:} This chapter provides a detailed review of the academic literature and technologies relevant to this work, with a research-focused analysis of \acp{llm}, \acp{crs}, and explainability in \ac{ai}.
    \item \textbf{Chapter 3 - Methodology:} This chapter presents the core research framework developed to address the research questions. It details the design of the conversation workflow, the user profiling model, the hybrid recommendation strategy, and the graph-based explanation generation method.
    \item \textbf{Chapter 4 - Experiments and Results:} This chapter describes the experimental setup, including the datasets and evaluation protocols. It then presents and analyzes the results from the comprehensive usability study conducted to evaluate the platform.
    \item \textbf{Chapter 5 - Conclusions and Future Work:} The final chapter summarizes the key findings of the research, discusses their implications in relation to the initial research questions, acknowledges limitations, and proposes potential directions for future work.
\end{compactitem}