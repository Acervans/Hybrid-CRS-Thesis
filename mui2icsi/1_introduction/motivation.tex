The application of \acp{llm} to \aclp{rs} is a novel and rapidly evolving field of research. While early \acp{crs} often relied on rigid, rule-based systems \cite{SOTA-CRS}, the fluidity of \acp{llm} allows for a much richer dialogue. However, the use of \acp{llm} in this domain is still a relatively underexplored area with significant challenges \cite{SOTA-CRS-LLM, BOOK:RS-HANDBOOK}. Modern techniques such as \ac{rag} and Function Calling are often employed to contextualize the model's responses and connect them to external tools.

A pivotal research gap lies in how to effectively translate a free-form conversation into a structured understanding of a user's tastes. This requires robust methods for dynamic user profiling and preference elicitation \cite[Conversational Preference Elicitation]{CHAPTER:RS-HANDBOOK-NLP}. Furthermore, as recommendation models become more complex, they often become ``black boxes'', making it difficult for users to understand why a particular item was suggested. This lack of transparency can hinder user trust and adoption. The field of \ac{xai} has become paramount in addressing this issue, with a strong focus on generating justifications for model outputs, a task for which \acp{llm} are particularly well-suited \cite[Generating Textual Explanations]{CHAPTER:RS-HANDBOOK-NLP} \cite{SOTA-RECSYS-EXPLAIN}.

This thesis is motivated by the need to address these specific research challenges. The work focuses on designing and evaluating a system that not only converses naturally but also actively builds a user profile from the dialogue and explains its recommendations.