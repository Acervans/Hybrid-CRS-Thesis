Si bien los \textit{\acp{llm}} han creado nuevos paradigmas para la interacción persona-ordenador, su aplicación en los Sistemas de Recomendación presenta importantes desafíos de investigación. Los sistemas tradicionales a menudo no logran capturar los matices de las preferencias del usuario, y la naturaleza de ``caja negra'' de muchos modelos erosiona su confianza. Se necesitan de Sistemas Conversacionales capaces de obtener preferencias del usuario dinámicamente, construir perfiles de usuario completos y ofrecer recomendaciones transparentes y explicables.

Este Trabajo de Fin de Máster aborda estos desafíos proponiendo y evaluando un marco híbrido para la recomendación conversacional explicable. El principal objetivo de investigación es indagar cómo una combinación de interacción conversacional, perfilado dinámico del usuario y razonamiento basado en grafos puede conducir a recomendaciones más eficaces, transparentes y centradas en el usuario.

Para ello, se diseñó un Sistema de Recomendación Conversacional híbrido, cuyo flujo conversacional es orquestado por un \textit{workflow} de LlamaIndex. Durante la interacción, se construye un perfil de usuario dinámico capturando preferencias explícitas e implícitas. Este perfil informa una estrategia de recomendación híbrida que utiliza tanto un motor basado en grafos en tiempo real usando \texttt{FalkorDB} para recomendaciones contextuales y explicables, como un modelo experto preentrenado (\texttt{EASER}) para una personalización profunda. Una contribución de este trabajo es el uso de razonamiento basado en grafos para generar explicaciones en lenguaje natural que justifiquen las sugerencias del sistema. Además, la plataforma fue evaluada empíricamente mediante un estudio de usabilidad con 11 participantes para medir la precisión de las recomendaciones, la satisfacción del usuario, el éxito en las tareas y la usabilidad percibida a través de la \textit{\ac{sus}}.

Los resultados son prometedores e indican la viabilidad del enfoque híbrido propuesto. El sistema alcanzó una puntuación \acs{sus} de 92 sobre 100, y los comentarios cualitativos de los usuarios destacaron la intuitividad de la interacción conversacional y la relevancia de las recomendaciones. Este TFM aporta un marco validado para la construcción de agentes de recomendación explicables, demostrando que la integración del perfilado dinámico del usuario y la generación de explicaciones basadas en grafos es un paso hacia la creación de Sistemas de Recomendación más fiables y eficaces.

\palabrasclave{Sistema de Recomendación Conversacional, IA Explicable, Perfilado de Usuario, Modelos Extensos de Lenguaje, Interacción Persona-Ordenador, Recomendación Basada en Grafos, Tests de Usabilidad}
