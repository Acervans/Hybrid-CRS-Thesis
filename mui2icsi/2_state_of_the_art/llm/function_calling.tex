Function Calling, also known as Tool Calling, is a mechanism that enables an \ac{llm} to interact with external tools and \acp{api}. This capability transforms the \ac{llm} from a passive text generator into an active agent capable of performing actions in the digital world. During an inference step, the model can decide that it needs to call an external function to fulfill a user's request. It then generates a structured JSON object containing the name of the function to call and the arguments to pass. The application code executes this function, and the result is fed back to the \ac{llm}, which uses it to generate its final response to the user. This technique is fundamental for creating agents that can interact with databases, execute code, or, as in the case of this project, update user preferences and invoke recommender models.