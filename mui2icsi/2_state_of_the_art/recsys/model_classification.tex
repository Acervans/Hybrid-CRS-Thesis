Recommendation algorithms can be broadly categorized into several families, each with distinct approaches to modeling user preferences.

\begin{compactitem}[\textbullet]
    \item \textbf{\ac{cf}:} This is the most prevalent paradigm in \ac{rs}. It operates on the principle of homophily, or ``wisdom of the crowd'', by identifying users with similar tastes or items with similar interaction patterns. \ac{cf} methods are typically divided into two sub-categories: \textit{memory-based} approaches, such as the widely-used Item-based k-Nearest Neighbors (\texttt{ItemKNN}) algorithm \cite{ITEMKNN}, and \textit{model-based} approaches, which learn latent factor representations of users and items. Latent factor models, such as those based on Matrix Factorization or \ac{fm}, are powerful but can suffer from the cold-start problem when new users or items are introduced. This has been addressed by applying an incremental algorithm to avoid having to compute the entire matrix from scratch \cite{COLD-START-INCREMENTAL-FM}. Recent approaches such as A-LLMRec demonstrate how \acp{llm} can be coupled with collaborative filtering knowledge to achieve strong performance across both cold and warm scenarios \cite{SOTA-LLM-CF}

    \item \textbf{Content-Based Filtering:} In contrast to \ac{cf}, content-based methods recommend items based on their intrinsic properties (e.g., genre, brand, textual description). The system learns a profile of the user's interests based on the features of items they have previously liked and recommends new items with similar features.

    \item \textbf{Hybrid Models:} Most modern, production-grade systems are hybrid, combining multiple recommendation strategies to leverage their respective strengths and mitigate their weaknesses. A common hybrid approach is to combine collaborative and content-based signals to improve recommendation quality, particularly for cold-start scenarios.

    \item \textbf{Deep Learning-Based Models:} More recently, deep learning has had a significant impact on the field. Graph Neural Networks (GNNs), for example, have become a powerful tool for learning complex, high-order relationships directly from the user-item interaction graph. Another impactful class of models is based on autoencoders, with models like \texttt{EASER} demonstrating strong performance on sparse data by learning a dense representation of the item space \cite{EASER}.
\end{compactitem}