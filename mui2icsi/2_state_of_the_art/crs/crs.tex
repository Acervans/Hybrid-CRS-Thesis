\aclp{crs} represent a paradigm shift in the field of recommendation, moving from the static, one-shot presentation of item lists to a dynamic, multi-turn dialogue between the user and the system. This evolution is motivated by the inherent shortcomings of traditional \acp{rs}, which struggle with unreliable preference estimation from sparse historical data, an inability to adapt to the user's immediate context, and the flawed assumption that users always have well-defined goals \cite{SOTA-CRS-SURVEY}. A \ac{crs} addresses these issues by engaging the user in a conversation to actively elicit their current needs and preferences, thereby providing a natural solution to the cold-start problem and enabling more accurate, context-aware recommendations \cite{SOTA-TOWARDS-CRS}.

To provide a structured understanding of this domain, this section first deconstructs a typical \ac{crs} into its core architectural components, and it then explores the latest methodological developments and research trends in the field.