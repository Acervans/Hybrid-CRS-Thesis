This research work sets out to investigate and develop a framework for a hybrid \ac{crs} that prioritizes dynamic user profiling and transparent, explainable recommendations. The primary contribution of this thesis is the successful implementation and user-centric evaluation of a novel system that demonstrates the viability of combining conversational \ac{ai} with graph-based reasoning and expert recommender models. The findings of the experimental evaluation provide answers to the core research questions posed at the outset of this document.

Regarding \textbf{RQ1}, the study demonstrated that a dynamic user profile, constructed in real-time from natural language conversation, is an effective method for improving personalization. The high user satisfaction ratings and low task difficulty scores from the usability study indicate that the conversational preference elicitation process was both intuitive and successful in capturing user needs.

In response to \textbf{RQ2}, the results affirm that orchestrating a hybrid recommendation strategy is an effective approach. The statistically significant improvement in perceived recommendation accuracy after retraining the expert model ($p < 0.05$) validates the benefit of integrating a deep, personalized model for established users, while the high contextual relevance scores in the cold-start phase demonstrate the effectiveness of the graph-based engine.

Finally, concerning \textbf{RQ3}, the research shows that leveraging graph-based reasoning to generate natural language explanations has a significant positive impact on the user experience. The explanations received exceptionally high scores for transparency and trust, and a relatively low score for cognitive load, suggesting that users found them clear, persuasive, and easy to understand. This solution effectively proves that graph-based reasoning can be applied even to black box \aclp{rs}, enhancing user trust and understanding without necessarily compromising the system's performance.

While the study's findings are promising, it is important to acknowledge its limitations. The usability study was conducted with a small and relatively homogeneous group of technologically experienced participants. A broader study would be needed to generalize these findings to a wider population. Furthermore, the evaluation focused on user-perceived quality rather than offline quantitative metrics.

The project repository with the code for the \texttt{HybridCRS} web platform, detailed in the complementary engineering thesis \cite{MUII-THESIS}, can be found at \url{https://github.com/Acervans/Hybrid-CRS}.